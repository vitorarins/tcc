\begin{abstract}

Currently many applications on the Internet follow the policy of keeping
some data accessible to the public. In order to do this, it's
necessary to develop a portal that is robust enough to ensure that all
people can access this data. But the requests made to recover
public data may not always come from a human. Companies
specializing in Big data have a great interest in data from public
sources in order to make analysis and forecasts from current 
data. With this interest, Web Crawlers are implemented. They are
responsible for querying data sources thousands of times a day,
making several requests to a website. This website may not be
prepared for such a great volume of inquiries in a short period of
time. In order to prevent queries to be made by
computer programs, institutions that keep public data
invest in tools called CAPTCHA (Completely Automated Public
 Turing test to tell Computers and Humans Apart). These tools usually deal
with images containing text and the user must enter what he or she
sees in the image. The objective of the proposed work is to perform the
text recognition in CAPTCHA images through the application of
convolutional neural networks.

\end{abstract}
