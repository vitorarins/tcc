\begin{resumo}
        
Atualmente, muitas aplicações na Internet seguem a política de manter
alguns dados acessíveis ao público. Para isso é necessário desenvolver
um portal que seja robusto o suficiente para garantir que todas as
pessoas possam acessá-lo. Porém, as requisições feitas para recuperar
dados públicos nem sempre vêm de um ser humano. Empresas
especializadas em Big data possuem um grande interesse em fontes de
dados públicos para poder fazer análises e previsões a partir de dados
atuais. Com esse interesse, \textit{Web Crawlers} são
implementados. Eles são responsáveis por consultar fontes de dados
milhares de vezes ao dia, fazendo diversas requisições a um
\textit{website}. Tal \textit{website} pode não estar preparado para
um volume de consultas tão grande em um período tão curto de
tempo. Com o intuito de impedir que sejam feitas consultas por
programas de computador, as instituições que mantêm dados públicos
investem em ferramentas chamadas CAPTCHA (teste de Turing público
completamente automatizado, para diferenciação entre computadores e
humanos). Essas ferramentas geralmente se tratam de imagens contendo
um texto qualquer e o usuário deve digitar o que vê na imagem. O
objetivo do trabalho proposto é realizar o reconhecimento de texto em
imagens de CAPTCHA através da aplicação de redes neurais
convolucionais.

\end{resumo}
