\chapter{Introdução}

Redes neurais artificiais clássicas existem desde os anos 60, como fórmulas  
matemáticas e algorítimos. Atualmente os programas de aprendizado de máquina  
contam com diferentes tipos de redes neurais. Um tipo de rede neural muito  
utilizado para processamento de imagens é a rede neural convolucional de  
profundidade. O trabalho em questão tratará da utilização e
configuração de uma rede neural convolucional de profundidade para
reconhecimento de textos em imagens específicas de CAPTCHAs.

\section{Problema}

Com o aumento constante na quantidade de informações geradas e 
computadas atualmente, percebe-se o surgimento de uma necessidade de tornar
alguns tipos de dados acessíveis a um público maior. A fim de gerar
conhecimento, muitas instituições desenvolvem portais de acesso para
consulta de dados relevantes a cada pessoa. Esses portais, em forma de
aplicações na Internet, precisam estar preparados para receber
diversas requisições e em diferentes volumes ao longo do tempo.

Devido a popularização de ferramentas e aplicações especializadas em Big
data, empresas de tecnologia demonstram interesse em recuperar grandes
volumes de dados de diferentes fontes públicas. Para a captura de tais
dados, Web crawlers são geralmente implementados para a realização de
várias consultas em aplicações que disponibilizam dados públicos.

Para tentar manter a integridade da aplicação, as organizações que possuem  
estas informações requisitadas investem em ferramentas chamadas CAPTCHA  
(teste de Turing público completamente automatizado para diferenciação entre  
computadores e humanos). Essas ferramentas frequentemente se tratam de  
imagens contendo um texto qualquer e o usuário precisa digitar o que vê na  
imagem. 

O trabalho de conclusão de curso proposto tem a intenção de retratar a  
ineficiência de algumas ferramentas de CAPTCHA, mostrando como redes neurais  
convolucionais podem ser aplicadas em imagens a fim de reconhecer o texto  
contido nestas imagens. 

\section{Objetivos}

\subsection{Objetivo geral}

Analisar o treinamento e aplicação de redes neurais convolucionais de
profundidade para o reconhecimento de texto em imagens de CAPTCHA.

\subsection{Objetivos específicos}

\begin{itemize}
        \item Estudar trabalhos correlatos e analisar o estado da arte;
	\item Entender como funciona cada aspecto na configuração de
          uma rede neural convolucional;
	\item Realizar o treinamento e aplicação de uma rede neural
          artificial para reconhecimento de CAPTCHAs.
\end{itemize}

\section{Escopo do trabalho}

O escopo deste trabalho inclui o estudo e análise de uma rede neural
convolucional de profundidade para reconhecimento de texto em imagens de 
um CAPTCHA específico.

Não está no escopo do trabalho:

\begin{itemize}
  \item Analisar outras formas de inteligência no reconhecimento de
    texto. 
  \item O estudo, análise ou implementação da aplicação de redes
    neurais convolucionais para outros tipos de problemas. 
  \item O estudo, análise ou implementação de softwares do tipo
    ``crawler'' ou qualquer programa automatizado para recuperar
    quaisquer informações de websites públicos.
  \item A análise e comparação de diferentes técnicas ou parâmetros
    para otimização de redes neurais.
\end{itemize}

\section{Metodologia}

Para realizar o proposto, foram feitas pesquisas em base de dados tais
como IEE Xplorer e ACM Portal. Adquirindo assim maior conhecimento
sobre o tema, estudando trabalhos relacionados. 

Com base no estudo do estado da arte, foram feitas pesquisas e
estudos para indicar caminhos possíveis para desenvolvimento da
proposta de trabalho.

\section{Estrutura do trabalho}

Para uma melhor compreensão e separação dos conteúdos, este trabalho
está organizado em 6 capítulos. Sendo este o capítulo 1 cobrindo a
introdução ao tema, citando os objetivos e explicando a proposta.

O capítulo 2 apresenta a fundamentação teórica, com as definições das
abordagens de desenvolvimento de aprendizado de máquina e redes
neurais. Os conceitos de tipos de redes neurais.

No capítulo 3 está a proposta de experimento a ser realizado. Assim
como uma breve ideia dos resultados esperados e a forma de avaliação
dos mesmos.

O capítulo 4 contém as informações do desenvolvimento do sistema de
reconhecimento de imagens de CAPTCHA. Também a apresentação dos dados
obtidos através das metodologias escolhidas na seção anterior.

No capítulo 5 são apresentados os resultados da aplicação do sistema
de reconhecimento de imagens de CAPTCHA.

Por fim, no capítulo 6 estão as conclusões obtidas através dos
resultados deste trabalho, as ameaças que podem comprometer o acesso à
dados públicos disponbilizados e as sugestões para trabalhos futuros
relacionados.